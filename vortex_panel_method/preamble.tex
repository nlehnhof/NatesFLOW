%\usepackage[utf8]{inputenc}

%graphics packages
\usepackage[]{graphicx}
\graphicspath{{./figures/}}

% ------------------ tikz ------------
%%tikz
\RequirePackage{pgf,tikz}
\usepackage{pgfplots}
\pgfplotsset{compat=newest}
\usetikzlibrary{arrows.meta}
\usetikzlibrary{arrows}
\usetikzlibrary{shapes.geometric}
\usetikzlibrary{shapes.misc}
\usetikzlibrary{backgrounds}
\usetikzlibrary{bending}
\usetikzlibrary{shadows}
\usetikzlibrary{patterns}
\usetikzlibrary{patterns.meta}
\usetikzlibrary{intersections}
\usepgfplotslibrary{patchplots}
\usepgfplotslibrary{fillbetween}
\usepgfplotslibrary{groupplots}
\usepgfplotslibrary{polar}
\usepgfplotslibrary{smithchart}
\usepgfplotslibrary{statistics}
\usepgfplotslibrary{dateplot}
\usepgfplotslibrary{ternary}
\pgfplotsset{%
	layers/standard/.define layer set={%
		background,axis background,axis grid,axis ticks,axis lines,axis tick labels,pre main,main,axis descriptions,axis foreground%
	}{grid style= {/pgfplots/on layer=axis grid},%
		tick style= {/pgfplots/on layer=axis ticks},%
		axis line style= {/pgfplots/on layer=axis lines},%
		label style= {/pgfplots/on layer=axis descriptions},%
		legend style= {/pgfplots/on layer=axis descriptions},%
		title style= {/pgfplots/on layer=axis descriptions},%
		colorbar style= {/pgfplots/on layer=axis descriptions},%
		ticklabel style= {/pgfplots/on layer=axis tick labels},%
		axis background@ style={/pgfplots/on layer=axis background},%
		3d box foreground style={/pgfplots/on layer=axis foreground},%
	},
}
% new style at automates partial ellipse
\tikzset{
    partial ellipse/.style args={#1:#2:#3}{
        insert path={+ (#1:#3) arc (#1:#2:#3)}
    }
}

\usepackage{caption}
\usepackage{subcaption}
\usepackage{placeins}
\usepackage{wrapfig}
\usepackage{url}

%tables
\usepackage{array}
\newcommand{\PreserveBackslash}[1]{\let\temp=\\#1\let\\=\temp}
\newcolumntype{C}[1]{>{\PreserveBackslash\centering}m{#1}}
\newcolumntype{R}[1]{>{\PreserveBackslash\raggedleft}m{#1}}
\newcolumntype{L}[1]{>{\PreserveBackslash\raggedright}m{#1}}

%mathpackages
\usepackage{amsmath}
%\usepackage{siunitx}
%\usepackage{mathrsfs}
%\newcommand{\vect}{\overset{\rightharpoonup}}
%decrease overset height
% \makeatletter
% \newcommand{\oset}[3][0.25ex]{%
% 	\mathrel{\mathop{#3}\limits^{
% 			\vbox to#1{\kern-0.5\ex@
% 				\hbox{$\scriptstyle#2$}\vss}}}}
% \makeatother

%%Vector arrow over variable
%\newcommand{\vect}[1]{%
%	\oset{\rightharpoonup}{#1}}

\RequirePackage{bm}
\newcommand{\vect}[1]{\bm{#1}}

\newcommand{\mat}[1]{%
	\mathbf{#1}}
\usepackage{nicefrac}
\usepackage{cancel}

%referencing packages
\usepackage{hyperref}
\usepackage[]{cleveref}

% ---------- colors -------------
\RequirePackage{xcolor}
\definecolor{navy}{HTML}{002E5D}
\definecolor{royal}{HTML}{005CAB}	% royal that matches BYU Engineering logo
\definecolor{darkgray}{HTML}{141414}
\definecolor{mediumgray}{HTML}{666666}	% medium gray definition from A. Ning
\definecolor{black}{HTML}{111111}
\definecolor{primary}{HTML}{005CAB}
\definecolor{secondary}{HTML}{c05367}
\definecolor{tertiary}{HTML}{8fa651}
\definecolor{plotsgray}{HTML}{808080}


% -------------- easy coloring of things ----------------
\newcommand{\navy}[1]{{\color{navy}#1}}
\newcommand{\primary}[1]{{\color{primary}#1}}
\newcommand{\secondary}[1]{{\color{secondary}#1}}
\newcommand{\tertiary}[1]{{\color{tertiary}#1}}
\newcommand{\gray}[1]{{\color{gray}#1}}

% Allow same footnotes
\newcommand*\samethanks[1][\value{footnote}]{\footnotemark[#1]}

% make \noindent where easier
\newcommand{\where}{\noindent where }
